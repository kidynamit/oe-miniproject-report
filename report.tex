%%%%%%%%%%%%%%%%%%%%%%%%%%%%%%%%%%%%%%%%%%%%%%%%%%%%%%%%%%%%%%%%%
%%%		1 Aug 2014 (revised)                                      %%% 
%%%  see also                                                                             %%%
%%% http://ravirao.wordpress.com/2005/11/19/latex-tips-to-meet-publication-page-limits/  
%%%%%%%%%%%%%%%%%%%%%%%%%%%%%%%%%%%%%%%%%%%%%%%%%%%%%%%%%%%%%%%%%

\documentclass[11pt,a4paper]{article}
\usepackage{times}
\usepackage{fancyhdr}           % Allows better control over headers and footers
%\usepackage{layout}            % use with \layout to see the page layout for
%debugging purposes.
\usepackage[margin=2.5cm]{geometry}  %   set the margins using the
                                %   geometry package (which is much
                                %   the easiest way of doing this).
\usepackage[pdftex]{graphicx}   %   Pictures (means you have to
                                %   produce pdf output via pdflatex)
\usepackage[small,compact]{titlesec}   % Try to reduce the white space
                                % latex loves so much
\titlelabel{\thetitle. \quad}   % Reduce space around section heads
                                % and add a full stop after the number
\pagestyle{fancy}               % Invoke fancy headers

\renewcommand{\abstractname}{\vskip -5mm}  %  Change name of Abstract
                                %  to nothing and loose some of the
                                %  excessive white space
\begin{document}

\title{Ontology Verbalisation in isiZulu:\\ A Web Application Approach} \date{}
\author{Ngoni Choga\\ nickchoga@gmail.com
\and Ntokozo Zwane\\ ntkzwane@gmail.com
\and Paul Wanjohi\\ wnjpau001@alumni.uct.ac.za}

%%%  Set the headers via fancyhdr package
\lhead{OE Miniproject Report} % Short title for running head
\chead{}
\rhead{\date{}}   %  Fixed running head of the date
\lfoot{}
\cfoot{\thepage}    %  add page number as centre footer.
\rfoot{}
\renewcommand{\headrulewidth}{0.0pt}   % Don't want horizontal line
                                % under header.

\maketitle
\thispagestyle{plain}  % First page is plain style headings and
                       % footers (ie just the page number as footer).

\begin{abstract}
	Abstract here.
\end{abstract}

\section{Approach}


\subsection{Introduction}
\label{ss:introduction}

South Africa is a country that is known and celebrated for its diversity. 
With eleven official languages, South Africa provides a unique challenge 
for ontologists in particular when it comes to verbalising ontologies.
Chavula and Keet mention that most ontologies available are in 
English in particular the name of the ontology elements used. \cite{RefWorks:32}

Ontology verbalisation involves the construction of understandable sentences 
in a natural language. This implies that some translation and processing must
occur to convert an ontology from it ontology language to a natural language. 
As a result, ontology verbalisation would be classified into the natural 
language generation class of problems, a less notable subset of natural
language process than natural language understanding. \cite{RefWorks:29} 

Since its inception, the Web Ontology Language (OWL) has had a lot research 
activity in the past decade that is geared towards allowing human users to 
view and edit OWL with much ease than dealing with raw code. Solutions 
constructed involve, the Manchester OWL syntax \cite{RefWorks:36}
and the graphical user interface Protege. \cite{RefWorks:34}
/
The need for localisation of these ontologies becomes more relevant especially
when looking ontology users such as domain experts who seek to view and edit
these ontologies. A challenge still faced among common ontology languages. \cite{RefWorks:33}

Attempts have been made to construct a solution to cater for specific natural 
languages. An example would include Attempto, which seeked to convert OWL 1.1 
ontologies to a controlled natural language Attempto Controlled English (ACE).
\cite{RefWorks:33}

Another notable example would be efforts made by Keet and Langa who were able
to create algorithms that would assist in the construction for an isiZulu 
grammar engine. They focus on the verbalisation of basic constructs need in
create an isiZulu grammar. \cite{RefWorks:35}

In this paper, a review of research material into this problem will be made. 
Furthermore, algorithms selected for ontology verbalisation will be reviewed and
implemented. Finally, an inspection of the challenges encountered will be reviewed
and conclusions will be drawn in the end.

\bibliography{references}
\bibliographystyle{plain}

\end{document}
